\title{Move-to-front transform}
\author{Vicente González Ruiz}
\maketitle
\tableofcontents

\section{Basics}
\begin{itemize}
\item
  MTF transform~\cite{manzini2001analysis} performs a change in the
  representation of the symbols of a sequence, where those symbols
  that have a high probability of occurrency are ``moved'' in the
  source alphabet towards decreasing positions.  Therefore, MTFT
  inputs \(N\) symbols and output \(N\) symbols.
\item
  The output sequence has a probability density function which follows
  an exponential distribution with a slope \(\lambda\):

  \begin{equation}
    f(x) = \left\{
      \begin{array}{ll}
        \lambda e^{-\lambda x} & \mbox{if $x \geq 0$};\\
        0 & \mbox{if $x < 0$}.
      \end{array} \right.
  \end{equation}
  \svgfig{graphics/exponential}{6cm}{600}
  where \(\lambda\) depends on the probability of ocurrence of the input
  symbols.
\end{itemize}

\section{Forward transform}
\begin{enumerate}
\def\labelenumi{\arabic{enumi}.}
\tightlist
\item
  Create a list \(L\) with the symbols of the source alphabet, where
  \[L[s]\leftarrow s; 0\le s\le r.\]
\item
  While the input is not exhausted:
  \begin{enumerate}
  \def\labelenumii{\arabic{enumii}.}
  \tightlist
  \item
    \(s\leftarrow\) next input symbol.
  \item
    \(c\leftarrow\) position of \(s\) in \(L\) (\(L[c]=s\)).
  \item
    Output \(\leftarrow c\).
  \item
    Move \(s\) to the front of \(L\).
  \end{enumerate}
\end{enumerate}

\subsection{Example}
\svgfig{graphics/MTF_example}{3cm}{300}

\section{Inverse transform}
\begin{enumerate}
\def\labelenumi{\arabic{enumi}.}
\tightlist
\item
  The step 1 of the forward transform.
\item
  While the input is not exausted:
  \begin{enumerate}
  \def\labelenumii{\arabic{enumii}.}
  \tightlist
  \item
    \(c\leftarrow\) next input code.
  \item
    \(s\leftarrow L[c]\).
  \item
    Output \(s\).
  \item
    The step 2.C of the forward transform.
  \end{enumerate}
\end{enumerate}

\subsection{Example}
TO-DO

\section{Lab}
TO-DO

\bibliography{text-compression}
